\documentclass[aps,prl,10pt,twocolumn,floatfix]{revtex4-2}
\usepackage{graphicx}
\usepackage{dcolumn}
\newcolumntype{d}{D{.}{.}{-1}}
\usepackage{url}
\usepackage{physics}

\bibliographystyle{apsrev4-2}


\begin{document}

%\begin{abstract}
%\end{abstract}


\title{Measuring the Speed of Light Using a Rotating Mirror}
\author{C. T. Rochelle}
\email{ctr233@msstate.edu}
\author{K. J. Grimes}
\date{\today}
\affiliation{Department of Physics and Astronomy\\Mississippi State University\\Mississippi State, MS 39762-5167}
\date{\today}

\maketitle

\section{Introduction}\label{Intro}
% Lead Up to Creation
One of the most important constants in all domains of Physics is the Universal Gravitation constant. 
It is used to determine the orbits of planets and galaxies as well as interactions between the smallest particles in the universe.
Since every single object in the universe is subject to this fundamental force, its precise calculation is very important.
This has led to many physicists and scientists trying to obtain smaller uncertainties and higher precision


\section{Theory}\label{Theory}
% Derivation of Formula 


\section{Experiment}
% What did you do?

% Schematic

% Describe system 

% Describe taking data


\section{Data Analysis}
% Explain the Results 


\section{Conclusion}
% Conclusion 


%\begin{thebibliography}{9}
%\bibitem{light} \textit{The Speed of Light}, Las Cumbres Observatory. (May 4, 2022). \url{https://lco.global/spacebook/light/speed-light/#:~:text=Galileo\%20concluded\%20that\%20the\%20speed,one\%20person\%20to\%20the\%20other.}.
%\bibitem{engineer} C. McFadden, \textit{Physics in a Nutshell: A Brief History of the Speed of Light}, Interesting Engineering. (Apr 25, 2017). \url{https://interestingengineering.com/a-brief-history-of-the-speed-of-light}.
%\bibitem{NIST} \textit{speed of light in vacuum} (National Institute of Standards and Technology, May 4, 2022).
%\end{thebibliography}

\end{document}
